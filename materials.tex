\chapter{Neutronics Support for Fusion Materials}

One of the ongoing tasks of this project was to provide \emph{ad hoc}
neutronics support to the fusion materials community.  Fusion energy systems
are an extremely challenging environment for materials, and their selection
and design must consider many factors, some of which are related to nuclear
analysis.  In many cases, this support played a role in guiding other
materials research without any specific outcomes or publications.  In a number
of cases, however, this project did produce unique outcomes.

\section{Nuclear Reponses in SiC Composites}

One such effort focused on understanding the nuclear responses in SiC/SiC
composite materials, in which the responses in the fiber, matrix, and
interface may be important.  Neutronics calculations have been performed to
determine the radiation damage parameters in the fiber, matrix, and interface
components of the SiC/SiC composite structural material employed in candidate
breeding blankets. The radiation damage parameters were calculated for both
the carbon and silicon sublattices. The breeder blanket concepts considered
included LiPb/SiC, Flibe/Be/SiC, and Li$_2$O/Be/He/SiC.

Nearly similar atomic displacement damage rates take place in Si and C. Helium
production in C is about a factor of 4 larger than that in Si. On the other
hand, significant hydrogen production occurs in Si with negligible amount in
C. As a result, the burnup of Si is about a factor of 2 more than that of
C. Property degradation depends on the kind of impurities introduced by
transmutations. The transmutation products include Al, Mg, Li, and Be. The
nonstoichiometric burnup of Si and C is expected to be worse than
stoichiometric burnups and could be an important issue for SiC. Damage
parameters were compared in candidate blanket concepts. We conclude that if
the dpa is the lifetime driver for SiC/SiC structure, the lifetime will be
significantly longer in Flibe/Be/SiC or Li$_2$O/Be/He/SiC \gls{FW/B} concepts
than in a LiPb/SiC blanket operating at the same neutron wall loading. On the
other hand, if gas production or burnup determine lifetime, the lifetime will
be slightly longer in a LiPb/SiC blanket.  In \gls{IFE} systems, the geometrical,
spectral, and temporal features influence the structure damage parameters. The
pulsed nature of \gls{IFE} systems results in very large instantaneous damage
parameters with the dpa and gas production being affected differently. It is
therefore essential to account for these unique features for accurate
prediction of the structure lifetime in \gls{IFE} systems.

The results given here provide an essential input for SiC/SiC composite
lifetime assessment. The impact of damage parameters on properties and
lifetime needs to be assessed. However, a determination of the effect of
fusion- neutron transmutations on the thermomechanical properties of SiC will
be required to set a lifetime for SiC components. It is speculated that the
lifetime will be set by swelling produced by transmuted
helium.\citeprod{sawan_radiation_2003}

\section{Assessing Applicability of Alternative Irradiation Environments for Fusion Materials Development}

Given the lack of appropriate irradiation facilities that specifically
represent a fusion energy environment, other irradiation facilities have been
considered for supporting fusion materials development.

Irradiation tests for candidate fusion structural and plasma facing materials
are usually performed in fission reactors such as \gls{HFIR} at \gls{ORNL}. To
understand how to correlate and extrapolate results from such tests to the
actual environment in fusion systems, we performed calculations to quantify
the damage parameters in the leading structural and plasma-facing armor
material candidates when used in \gls{MFE} and \gls{IFE} systems and when
irradiated in \gls{HFIR}. The structural materials considered are the ferritic
steel alloy F82H, austenitic steel SS316, the vanadium alloy V4Cr4Ti and the
SiC/SiC composite. Plasma-facing armor material candidates included Be, W, and
\gls{CFC}.  Atomic displacement damage and gas production rates are greatly
influenced by the neutron energy spectrum. The results indicate that for the
same neutron wall loading, atomic displacement damage is slightly lower in
\gls{IFE} systems than in \gls{MFE} systems but gas production is about a
factor of 2 lower due to the softened neutron spectrum. In addition, much
lower gas production is obtained in samples irradiated in fission
reactors. For the same atomic displacement damage level, gas production is
significantly lower in fission irradiation compared to that in the first wall
of a fusion blanket with the effect strongly dependent on the material. For
SS316, the high helium production in B and Ni by low energy neutrons yields
higher helium production following irradiation in fission reactors. The
results of this work will help guide irradiation experiments in fission
reactors to properly simulate the damage environment in fusion systems by
possible gas implantation and will facilitate extrapolating to the expected
material performance in fusion systems. In addition, the results represent a
necessary input for modeling activities aimed at understanding the expected
effects on mechanical and physical properties.\citeprod{sawan_damage_2012}

While similar analysis was desired for the irradiation environment of
\gls{SNS} at \gls{ORNL}, there is a lack of high quality data for the
important nuclear responses, namely DPA, gas production and transmutation
rates, to produce reliable estimates.







