\chapter{Nuclear Data for Fusion Applications}

The nuclear data community is constantly seeking to improve the data that is
available for nuclear analysis, particulary the neutron transport and
transmutation cross-sections.  This results in regular changes to the standard
data sets for all nuclear applications, some of which are adopted by the
standard data set that is targeted at fusion neutronics applications,
\gls{FENDL}.  Under this project, we monitor the developments of these nuclear
data sets and test any updates to understand the impact of those updates on
the predictions of nuclear responses in fusion energy systems.

During this project, there was a new release of the fundamental US-based data
set, ENDF/B-VII.0, some of which was incorporated in an update of the the
international fusion nuclear data set, FENDL-2.1.  Version 2.1 of \gls{FENDL}
includes 71 elements or isotopes.  Data for most of the isotopes/elements (40)
were taken from ENDF/B-VI.8. Following the release of ENDF/B-VII.0 we
performed MCNP calculations for a 1-D calculational benchmark representative
of an early ITER design that was utilized during the FENDL development
process. Calculations were carried out using FENDL-2.1 with the data for the
40 isotopes/elements replaced by the recent data from ENDF/B- VII.0 and the
results for flux, heating, dpa, and gas production were compared to those
obtained using the FENDL-2.1 library.

The results presented in this work for the ITER calculational benchmark
clearly show that the previously observed differences in nuclear heating and
radiation damage are removed when we use the recent correctly processed
ENDF/B-VII.0 data. Differences in all nuclear parameters are very small
implying that minimal impact is expected on ITER analysis and updating the
FENDL-2.1 library is not urgently needed for ITER analysis.

However, a larger effect is expected when used in analysis of other fusion
systems with breeding blankets (Demo and power plants). Calculations for an
inertial fusion power plant showed relatively large changes in nuclear
parameters due to the large changes in the H-3 and Au-197 data that affect the
energy spectrum of neutrons emitted from the ICF target. The results confirm
the need for updating FENDL-2.1 for use in analysis of fusion systems beyond
ITER.\citeprod{sawan_impact_2009, sawan_benchmarking_2009, sawan_impact_2011}

Similar benchmarking occurred with the release of \gls{FENDL}-3, showing a
modest increase of the neutron flux levels in the deep penetration regions and
a substantial increase in the gas production in steel components.  The
comparison to experimental results showed good agreement with no substantial
differences between FENDL-3.0 and FENDL-2.1 for most responses.  There is a
slight trend, however, for an increase of the fast neutron flux in the
shielding experiment and a decrease in the breeder mock-up experiments. The
photon flux spectra measured in the bulk shield and the tungsten experiments
are significantly better reproduced with FENDL-3.0 data. In general, FENDL-3,
as compared to FENDL-2.1, shows an improved performance for fusion neu-
tronics applications. It is thus recommended to ITER to replace FENDL-2.1 as
reference data li- brary for neutronics calculation by FENDL-3.0.\citeprod{fischer_benchmarking_2014}


