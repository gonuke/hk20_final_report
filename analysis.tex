\chapter{Nuclear Analysis of Fusion Energy Systems}

\section{\acrfull{APEX}}

This project began as part of the APEX study that had the goal of identifying
a first wall and blanket system that could withstand much higher power
densities than conventional system, while still meeting the shielding, cooling
and tritium breeding needs for viable fusion energy.  With neutron wall loads
>10 MW/m$^2$, surface heat fluxes of >2 MW/m$^2$ and thermal efficiencies
>40\%, such concepts were expected to greatly increase the economic
attractiveness of fusion energy.  These concepts varied from flowing
particulate systems, to flowing liquid first walls of various compositions, to
solid walls with high heat capacity fluids flowing near the first wall.

A typical nuclear analysis considers a range of important engineering
responses to the radiation flux, including tritium breeding, energy
multiplication, local heating and radiation damage in the blanket system
itself, and heating and damage to sensitive components such as magnets.  With
the addition of an activation calculation, it is possible to consider decay
heat during off-normal conditions, dose rates during maintenance, and waste
disposal ratings over the long term.

\subsection{\acrfull{APPLE}}

The first concept was a flowing LI$_2$O paticulate blanket concept without a
structural first wall.  Neutronics calculations were performed for the
\gls{APPLE} concept using a neutron wall loading of 10 MW/m$^2$ . Steel
structure was used in the shield and vacuum vessel. The Li$_2$O in the blanket
had a packing fraction of 0.6. A minimum blanket thickness of 40 cm was
required for the steel structure to be a lifetime component. The radial build
that satisfied requirements for steel structure lifetime, VV reweldability,
and superconductor magnet shielding was determined. The achievable overall TBR
was estimated to be >1.2 taking into account the difference in blanket
thickness in the regions surrounding the plasma. Although the steel structure
and VV were considered lifetime components, the SiC baffles and separation
walls, which are not structure members, experienced a large damage rate and
were designed for quick replacement. More than an order of magnitude reduction
in decay heat and activity resulted from placing the structure behind the
Li$_2$O particulate blanket. Using low activation ferritic steel structure
behind the Li$_2$O particulate blanket allowed for near surface burial of the
radwaste.  It was concluded that the Li$_2$O particulate blanket concept
without a structural first wall had the potential for achieving tritium
self-sufficiency with lifetime structure and reweldable vacuum vessel.  It had
attractive safety features resulting from the significant reduction in
radioactivity and decay heat generation in the structural
material.\citeprod{sawan_nuclear_2000}

\subsection{\acrfull{EVOLVE}}

Another early concept relied on the evaporation of liquid lithium at the first
wall to absorb the energy and protect the solid components.  The neutronics
performance of the EVOLVE concept was analyzed using two-dimensional
calculations. Using tungsten alloys yields \textapprox 10\% higher TBR than
with tantalum and was used in the reference design. The overall TBR with tray
zones having a radial thickness of 50 cm in the \gls{OB} region and 40 cm in
the \gls{IB} region is 1.37 at a lithium enrichment of 40\% $^6$Li. This value
assumes 75\% outboard blanket coverage, 15\% inboard blanket coverage, and no
breeding in the divertor region. Tritium breeding has a comfortable margin
that allows for design flexibility. The amount of nuclear heat deposited in
the different regions was determined. Most of nuclear heating (\textapprox
72\%) is deposited in the high temperature front blanket. Adding the surface
heat deposited in the \gls{FW} implies that \textapprox 76\% of the total
\gls{IB} and \gls{OB} energy is deposited as high-grade heat in the front
evaporation-cooled zone (\gls{FW} and trays) and carried by the Li vapor to
the heat exchanger. Since this heat is extracted at a temperature of 1200°C, a
thermal efficiency >60\% in the power conversion system can be achieved by
employing a closed cycle helium turbine system.

Nuclear heating and radiation damage profiles were calculated. No significant
poloidal peaking is observed. The peak damage rate in the \gls{OB} secondary
blanket and \gls{IB} shield is about a factor of \textapprox 6 lower than in
the \gls{FW} for which a damage rate of \textapprox 35 dpa/year has been
calculated. This implies that they are expected to have a factor of 6 longer
lifetime than the \gls{FW} and trays. The lifetime of the \gls{OB} shield is
about an order of magnitude longer than for the \gls{OB} secondary blanket and
the \gls{IB} shield making it a lifetime component. The radial build required
for VV reweldability and magnet shielding was
determined.\citeprod{sawan_neutronics_2001,mattas_evolveadvanced_2000,abdou_exploration_2001}

\subsection{Solid Wall with Flibe Coolant}

Another design option considered in the \gls{APEX} program featured a solid
\gls{NCF} alloy Flibe (Li$_2$BeF$_4$) acting as a combined breeder and
coolant.  Work done under this award included both the structural design of at
least two such concepts, as well as the analysis of neutronics performance.

The spiral blanket is an innovative design of a solid \gls{FW} concept
utilizing \gls{NCF} steel in combination with Flibe coolant. Thermal hydraulic
evaluation has shown that it is capable of dissipating an average neutron wall
loading of 6.4 MW/m$^2$, a peak neutron wall loading approaching 10 MW/m$^2$
and surface heating of 1.3 MW/m$^2$ , while satisfying the temperature limits
ascribed to NCF steel of 800$^o$C maximum and 700$^o$C interface with
Flibe. The pressure drop is a modest 0.56 MPa and the stresses for 12 YWT
\gls{NCF} steel are well within limits with time effects taken into
account. The overall \gls{TBR} is 1.33 using natural Li and the energy
multiplication is 1.26. The amount of tritium bred in the Be over the
estimated blanket lifetime is 1.5 kg. However, at the operating temperature of
the Be, most of the tritium will diffuse out. Fabrication of the blanket using
forming techniques and diffusion bonding appears to be possible. Using an
inlet Flibe temperature of 500$^o$C, an outlet from the primary blanket of
590$^o$C and after traversing the rear blanket, exiting at 600$^o$C, and
assuming a supercritical steam power cycle with double wall heat exchanger, an
efficiency of near 48\% can be expected. Finally, this unconventional
innovative design shows that solid wall blankets have a legitimate place along
with liquid protected walls as a first line of defense facing an intense
neutron plasma.\citeprod{sviatoslavsky_spiral_2003}

Another concept relies on a simpler geometric design, with a \gls{Pb} neutron
multiplier.  A description of the geometric and engineering aspects, and
preliminary stress analysis of the re-circulating blanket was developed. It
was shown that the time-dependent and time-independent primary stresses as
well as the primary and secondary stress limits were satisfied at all points
in the blanket. Material, and material compatibility issues were addressed and
solutions offered. Fabrication possibilities and coolant circuits were
presented. Even though the re-circulating solid wall blanket is somewhat
complicated, its aimed to maximize nuclear parameters to achieve high
conversion efficiency.\citeprod{sviatoslavsky_engineering_2003, sviatoslavsky_engineering_2004, wong_molten_2004}

Neutronics calculations were performed to determine the relevant nuclear
performance parameters for the blanket. With an enrichment of 40\% $^6$Li the
overall tritium breeding ratio is expected to be \textapprox 1.16 and the
blanket energy multiplication is 1.13. Nuclear heating profiles were
determined for the different components of the blanket and used in the thermal
hydraulics analysis. The peak structure dpa and helium production rates are
77.6 dpa/FPY and 955 He appm/FPY, respectively, implying a lifetime of
\textapprox 3 FPY. Calculations were performed to determine the radial build
in both the inboard and outboard regions required to provide adequate
shielding for the vacuum vessel and \gls{TF} magnets. The \gls{NCF} structure
dominates the total activity and decay heat. The Mo content in the \gls{NCF}
needs to be reduced from 0.02\% to $<$0.01\% for the structure waste to
qualify as low level class C waste.  While the \gls{WDR} of the Flibe is well
below unity, the \gls{Pb} has to be circulated at a very small flow rate ($<$1
cm$^3$/s) to remove the generated $^{208}$Bi and allow \gls{Pb} disposal as
low level waste.\citeprod{sawan_nuclear_2003}


\subsection{Tritium Breeding Ratio}

In comparing different potential molten salts, a key performance criteria is
the \gls{TBR}.  The molten salt LiF--NaF--BeF$_2$ (Flinabe) with ratio 1:1:1 has
been suggested as the \gls{FFLL} in the \gls{CLiFF} high power density
concept. Flinabe has intrinsic properties that favor its use as the
\gls{FFLL}. The concern raised with regard to its potential for tritium
breeding has been examined in this paper and comparison was made to the Flibe
(LiF:BeF$_2$ , with ratio 2:1) in two arrangements: (1) as a \gls{FFLL} and
breeder in the blanket; and (2) as only the \gls{FFLL} with LiPb/SiC
conventional blanket following the \gls{FW}. Two types of structure were also
considered, \gls{FS} and SiC.  In the first configuration and with \gls{FS}
structure, the local \gls{TBR} in Flinabe is less than in Flibe by \textapprox
15\% and by \textapprox 9\% at natural Li and 90\% Li-6 enrichment,
respectively. The \gls{TBR} maximizes at 50\% Li-6 (Flinabe) and at 25\% Li-6
(Flibe) with values \textapprox 1.04 and \textapprox 1.16, respectively. Using
SiC as the structure material reduced \gls{TBR} even further (\textapprox
8\%). A front beryllium multiplier must be introduced in the blanket for both
breeders.  Significant improvement in \gls{TBR} is achieved with the Be
zone. To achieve a \gls{TBR} of 1.4, the required Be zone thickness is
\textapprox 6 (Flibe) and \textapprox 10 cm (Flinabe) with SiC. This is
equivalent to an effective thickness \textapprox 4 (Flibe) and \textapprox 6
cm (Fli- nabe) of solid Be zone. If a conventional LiPb/SiC blanket is
employed, the local \gls{TBR} with Flinabe \gls{FFLL} is less than Flibe by
\textapprox 3\%. With this 2-cm thick layer, tritium self-sufficiency
(\gls{TBR} \textapprox 1.4) can be achieved with LiPb enriched to \textapprox
35\% Li-6 or higher. Thus, tritium self-sufficiency is not a feasibility issue
for Flinabe if used as a \gls{FFLL} and breeder (with Be multiplier) or as a
\gls{FFLL} only with enriched LiPb blanket. This was shown to be the case when
account is made to all sources of uncertainties in the achievable and required
\gls{TBR}.\citeprod{youssef_breeding_2002}

\subsection{Activation and waste disposal}

A more detailed study was performed to understand the impact of switching to
liquid walls on the volumes and classification of waste.  Structural waste
volume and hazard were compared in a thick \gls{LW} Li/V-4Cr-4Ti system with
maximum neutron wall load of 10 MW/m$^2$ and in a conventional \gls{SW} Li/V
blanket with maximum neutron wall load of 5 MW/m$^2$. The comparison was made
for two configurations, namely: (1) “Fixed Radii” where the \gls{LW} and
\gls{SW} \gls{FW/B} have the same plasma and FW radii, and (2) “Fixed Fusion
Power” with the \gls{LW} \gls{FW/B} having half the plasma and \gls{FW} radii
and hence twice the neutron wall load of the conventional \gls{SW}
\gls{FW/B}. Shield optimization was performed to satisfy the same acceptable
damage parameters in the magnet. The objectives were to quantify the advantage
of using the thick \gls{LW} blanket option over a conventional \gls{SW} system
in reducing disposed waste volume and its hazard.

The analysis indicated that the total waste volume from the machine (including
magnets and \gls{VV}) were dominated by waste from the shield (\textapprox
50-63\%). The \gls{FW/B} contributed \textapprox 22\% of the total waste
volume in the conventional \gls{SW} blanket and \textapprox 6\% in the two
\gls{LW} options. The structure waste volume per GW$_{th}$ was almost the same
for both \gls{LW} options. The waste volume per GW$_{th}$ of the conventional
\gls{SW} \gls{FW/B} was larger than that in both \gls{LW} options by a factor
of seven. However, the total waste volume per GW$_{th}$ in the conventional
\gls{SW} option was \textapprox 2.14 larger than the value in both \gls{LW}
options. Regarding \gls{WDR}, the results for the two \gls{LW} options were
identical. The \gls{WDR} values for the \gls{LW} and conventional \gls{SW}
concepts were comparable. All components were classified as Class C low level
waste. Values for activity and decay heat per MW$_{th}$ were comparable for
the \gls{LW} and conventional \gls{SW} blanket options for the permanent
components. However, values in the \gls{FW/B} zone of the \gls{SW} were much
higher due to the much larger structure
content.\citeprod{youssef_comparison_2002, youssef_component_2002, youssef_radwaste_2002}

In addition to waste disposal issues, another major concern with using Flibe
and Flinabe in fusion blankets is that transmutation of the constituent
elements leads to the production of the highly corrosive free F and the
relatively less corrosive TF. Controlling the activities of free F and TF is
essential for consideration as viable breeder/coolant candidates. Neutronics
calculations were performed for blanket designs using either Flibe or the low
melting point Flinabe to determine the transmutation rates of constituent
elements and the rates of production of other elements. The results showed
that at least from mass balance considerations no free F will be left provided
that the recombination reactions with freed Be, Li, Na, and produced tritium
are fast enough. However, more than 95\% of the tritium bred will be in the
form of TF. The results indicate also that O and N are produced at about 7\%
and 14\% of the rate of tritium breeding.  Enrichment of Li was found to have
minor impact on mass balance with the conclusions remaining the same. The
results imply that we only have to worry about chemistry control of the less
corrosive TF. Be can be used to reduce both TF and F$_2$ to BeF$_2$ and T$_2$.

The transmutation rates and the thermodynamics define the chemistry conditions
of the Flibe and Flinabe under neutron irradiation. These conditions need to
be reproduced in the experimental setup at INEEL to study the thermodynamics
and kinetics of the REDOX reaction.  After the REDOX condition is established,
the material corrosion experiment can be carried out to assess the
compatibility between the molten salt, with proper chemistry control, and
different structural materials.\citeprod{sawan_transmutation_2003}

\section{ITER \gls{TBM}}

A wide variety of activities related to the neutronics performance of a
potential ITER \gls{TBM} were pursued, ranging from high level consideration
of R\&D strategies and needs to detailed analysis of specific design options.

\subsection{Neutronics Testing in ITER}

One such high level assessment was a discussion of strategies and requirements
placed on the machine operation (fluence, wall load, etc.) and on the
\gls{TBM} geometrical arrangement (size, dimension, spatial locations) in
order to have a meaningful neutronics testing in ITER.  Neutronics tests
planned to be performed in ITER aim at resolving key neutronics issues such as
the demonstration of tritium self-sufficiency and verification of the adequacy
of calculation tools and nuclear data bases in accurately assessing the
nuclear environment.  Although each ITER partners will perform these tests
inside their designated \gls{TBM} that may utilize different breeders,
coolants, and structure, neutronics issues to be resolved are generic in
nature and are important to each TBM type. Dedicated neutronics tests
specifically address the accuracy involved in predicting key neutronics
parameters such as \gls{TPR}, volumetric heating rate, induced activation and
decay heat, and radiation damage to the reactor components. In addition,
neutronics analyses are required to provide input support for other tests
(e.g.  heating rates for thermo-mechanics tests). While tritium
self-sufficiency for a given blanket concept can only be demonstrated in a
full sector, as envisioned in a DEMO, including a closed tritium fuel cycle,
testing in ITER TBM can provide valuable information regarding the main
parameters needed to assess the feasibility of achieving tritium
self-sufficiency.

From instrumental consideration, all neutronics parameters, except induced
activation, can be performed in either low fluence (\textapprox 1 MWs/m$^2$)
or very low fluence (\textapprox 1 w.s/m$^2$) operation mode. The low fluence
level could be achieved, for example, with a wall load of 0.0025 MW/ m$^2$ and
400 s pulse. In ITER, the wall load at the \gls{TBM} is \textapprox 0.78 MW/
m$^2$ , which is much larger than the 0.0025 MW/ m$^2$ value. One pulse or two
is adequate for these tests.

Unlike the case of using engineering scaling to reproduce DEMO-relevant
parameters in an “act-alike” test module, dedicated neutronics tests require a
“look-alike” test module for a given blanket concept. Furthermore, measured
neutronics data requires a high spatial resolution. This necessitates the
measured quantity to be as flat as possible in the innermost locations inside
the test module. In the present work, we confirmed this requirement based on
results from two-dimensional calculations in an $R-\theta$ model that accounts
for the presence of the ITER shielding blanket and the surrounding frame of
the port where the US and Japan \gls{TBM}s are placed. It is shown that the
profiles of the \gls{TPR} and heating rates have flat values over a range of
10-20 cm in the toroidal direction and a range of \textapprox 1 cm in the
radial direction inside the breeding layers of the \gls{HCPB}
\gls{TBM}. \citeprod{youssef_strategy_2005, youssef_nuclear_2006}

\subsection{Tritium self-sufficiency}

In considering the design of an ITER \gls{TBM}, we identified the need for a
comprehensive review of the physics and technology conditions for attaining
tritium self-sufficiency for the D-T fuel cycle.  There is no practical
external source of tritium for fusion energy development beyond ITER, and all
subsequent fusion systems have to breed their own tritium. Tritium
self-sufficiency in DT fusion systems cannot be assured unless specific plasma
and technology conditions are met. We addressed these conditions and shed
light on a possible “phase-space” of plasma, nuclear, material, and
technological conditions in which tritium self-sufficiency can be attained.

It is crucial that the tritium fractional burnup in the plasma be kept high,
at least above a minimum of 2\% and most preferably above 5\%. Thus, plasma
edge physics modes that lead to higher tritium recycling need to be
explored. A reserve tritium inventory to keep fueling the plasma and continue
reactor operation during periods of malfunction of the tritium processing
system is necessary. To keep the reserve inventory, and hence the required
\gls{TBR}, sufficiently low requires high reliability/availability of the
tritium processing system, and redundancy in some of the tritium processing
system, especially the plasma exhaust processing line. Design options that
minimize tritium inventories in reactor components such as the blanket, FW,
and divertor are needed.

Up to 30\% reduction in \gls{TBR} could result from using 20\% structure in
the blanket. Hence, it is necessary to accurately determine the amount and
configuration of structure required to ensure structural integrity of the
blanket under normal and abnormal conditions. Practical FW thickness and
blanket structure content based on detailed structural-mechanical and
thermal-hydraulics analyses 15need to be well defined. Accurate definition of
other blanket design considerations that introduce uncertainties in the
\gls{TBR} (e.g., using separate coolant and/or neutron multiplier, and the
need for electric insulator) is necessary. Using stabilizing shells and
conducting coils for plasma control and attaining advanced plasma physics
modes should be examined carefully to minimize the impact on tritium
breeding. The size and materials used in plasma heating and current drive
components and fueling and exhaust penetrations impact the \gls{TBR}. Use of
strong neutron absorbers in these systems should be eliminated or minimized
and design options that minimize streaming path should be
considered. Calculation of the \gls{TBR} should be based on detailed 3-D
models that account for all design details. Neglecting heterogeneity effects
results in errors up to \textapprox 10\% in predicting the \gls{TBR}.
Integral experiments are needed to validate and improve nuclear data.

It is necessary to establish without delay an extensive parallel and highly
interactive R\&D program in plasma physics, plasma control technologies, plasma
chamber systems, materials science, safety, and systems analysis to determine
the “phase-space” of plasma, nuclear, material, and technological conditions
in which tritium self-sufficiency can be attained.\citeprod{sawan_physics_2006}


\subsection{Molten Salt Design Variations}

Early design concepts for a US ITER \gls{TBM} considered molten salts as
breeding coolants.  The design and analysis of the above specific concept led
to an interest in a number of different molten salt based design concepts.
Neutronics assessment was performed for a number of molten salt breeding
blanket concepts that could be utilized in fusion power plants. Special
attention was given to concepts that could be developed, qualified and tested
in the time frame of ITER. The conventional ferritic steel alloy F82H with a
temperature limit of 550$^o$C is considered as the structural material. The
concepts evaluated were a self-cooled Flinabe blanket with Be multiplier and
dual-coolant blankets with He-cooled FW and structure.  Several options were
considered for the dual-coolant concept, including using Be or Pb
multiplier. In addition, three different molten salts were considered
including the high melting point Flibe, a low melting point Flibe, and
Flinabe. Several iterations were made to determine the blanket radial build
that achieves adequate \gls{TBR}. Larger margins were considered to account
for uncertainties resulting from approximations in modeling.  The same
\gls{TBR} can be achieved with a thinner self-cooled blanket compared to the
dual-coolant blanket. A thicker Be zone is required in designs with
Flinabe. The overall \gls{TBR} will be \textapprox 1.07 based on 3-D
calculations and excluding breeding in the divertor region. Minor design
modifications can be made to enhance the \gls{TBR} if needed to ensure tritium
self-sufficiency. We concluded that the molten salt design concepts have the
potential for achieving tritium self-sufficiency. Using Be yields higher
blanket energy multiplication. A modest amount of tritium is produced in the
Be (\textapprox 3 kg) over the blanket lifetime of \textapprox 3 FPY. Using He
gas in the dual-coolant blanket resulted in about a factor of 2 lower blanket
shielding effectiveness. With a total blanket/shield/\gls{VV} radial build of
105 cm in the \gls{IB} and 120 cm in the \gls{OB} it was possible to ensure
that the shield would be a lifetime component, the \gls{VV} would be
reweldable, and the magnets would be adequately shielded. Based on this
analysis we concluded that molten salt blankets can be designed for fusion
power plants with neutronics requirements such as adequate tritium breeding
and shielding being satisfied.\citeprod{sawan_neutronics_2005}

\subsection{Expanded Liquid Breeder Assessment}

As candidate blankets for a U.S. Advanced Reactor Power Plant design, and in
consideration of the time frame for the ITER development, we assessed first
wall and blanket design concepts based on the use of \gls{RAFS} as structural
material and liquid breeder as the coolant and tritium breeder. The liquid
breeder choice includes the conventional molten salt Li$_2$BeF$_4$ and the low
melting point molten salts such as LiBeF$_3$ and FLiNaBe. Both self-cooled and
dual coolant molten salt options were evaluated. We have also included the
dual coolant lead-eutectic Pb-17Li design in our assessment. We based our
first wall and blanket assessment on an Advanced Reactor Power Plant design,
which has a maximum neutron wall loading of 5.4 MW/m$^2$ and a maximum surface
heat flux of 1 MW/m$^2$ at the outboard mid-plane of the tokamak
reactor. Molten salt blankets will require an additional neutron multiplier
like Be or Pb to provide adequate tritium breeding. For the dual coolant
design, helium is used to remove the first wall surface heat flux and to cool
the entire steel structure. The liquid breeder is circulated to external heat
exchangers to extract the heat from the breeding zone (a “self-cooled”
breeding zone). We take advantage of the molten salt low electrical and
thermal conductivity to minimize impacts from the \gls{MHD} effect and the
heat losses from the breeder to the actively cooled steel structure. For the
Pb-17Li breeder we employ \gls{FCI}s with a low electrical and thermal
conductivity to perform respective insulation functions. For the dual-cooled
molten salt blanket options, to avoid the formation of a thin solid layer of
high melting point molten salt in the blanket, the utilization of a lower
melting point molten salt is recommended.  For the lower melting point molten
salt FLiNaBe, physical properties like melting point will need to be further
established. For the molten salt designs, REDOX control will need to be
demonstrated to mitigate the compatibility issue between the generated
\gls{TF} and structural material. Due to the low tritium solubility of molten
salt and Pb-17Li, tritium permeation control will be needed to minimize
contamination of the environment. For the duel-coolant Pb-17Li design,
successful development of the \gls{FCI}, such as SiC/SiC composite, is
required to arrive at an acceptable \gls{MHD} pressure drop and to thermally
insulate the high temperature self-cooled breeder from the \gls{RAFS}
structure. Results of the above R\&D items will then form the technical basis
for the selection of the reference blanket concept for the U.S. and provide
input for the formulation of the U.S.\ ITER test module program and its
corresponding test plan.\citeprod{wong_assessment_2005, abdou_u.s._2005}

As part of this broader analysis, detailed 3-D neutronics calculations were
performed for the dual coolant molten salt blanket designs with the low
melting point Flibe or Flinabe in a tokamak power plant configuration. The
model included the detailed heterogeneous geometrical arrangement of the
blanket sectors. The Be multiplier zone thickness was 5 cm with Flibe and 8 cm
with Flinabe. The 3-D model included water-cooled steel with tungsten armor in
the divertor region. The total TBR was determined to be \textapprox 1.07.
This was a conservative estimate since it assumed no breeding in the double
null divertor zones on which 12\% of the source neutrons impinge. We
demonstrated that minor design modifications such as increasing the Be zone
thickness can be made to enhance the \gls{TBR} if needed to ensure tritium
self-sufficiency. We concluded that the dual coolant molten salt design
concept has the potential for achieving tritium self-sufficiency. The
calculated \gls{TBR} that accounts for heterogeneity and 3-D geometrical
effects was \textapprox 6\% lower than estimates based on 1-D
calculations. The 1-D calculations tended to overestimate nuclear heating in
the blanket by \textapprox 8\% resulting in overestimating the plant thermal
power. The peak dpa and helium production rates in the structure were 28
dpa/FPY and 356 appm/FPY, respectively, and occured in the \gls{OB} blanket at
mid-plane. Comparing the 3-D results with the 1-D results indicated that the
approximate 1-D calculations overestimated damage and nuclear heating in the
\gls{FW} and front zone of the blanket by factors of 1.3-1.7. However, the 1-D
calculations resulted in a steeper radial drop in nuclear parameters leading
in significant underestimation (by up to a factor of 3) of radiation effects
at the back of the blanket. When combined with peaking factors of up to
\textapprox 3 obtained due to the 3-D geometrical heterogeneity effects, it
was concluded that 1-D calculations significantly underestimated radiation
damage in the shield and vacuum vessel behind the blanket and large design
margins should be allowed when 1-D calculations are used in shielding
assessment.\citeprod{sawan_three-dimensional_2006}

An activation analysis was also performed for these two coolants.  Two
\gls{DC} blanket concepts were studied by the US for Demo reactor in which
helium is used to cool the \gls{FW} and structure whereas molten salt is used
as both coolant and breeder. The F82H conventional steel was used as the
structural material. The low melting point “Flibe” (\textapprox 380 $^o$C) was
used in the DC/Flibe option while the “Flinabe” (\textapprox 305 $^o$C), was
used in the DC/Flinabe option. The blanket in both concepts had a thickness of
65 cm in the \gls{OB} (40 cm in \gls{IB}), including a 5 cm-thick front Be
zone (8 cm in case of Flinabe). The \gls{FW} was estimated to last \textapprox
5 full power years before replacing the blanket. The radioactivity and decay
heat, after shutdown, were assessed separately for the structural material,
the Be multiplier and the breeder (Flibe/Flinabe). The total activity and
decay heat in the F82H structure was very similar in both concepts. The Flibe
activity in the DC/Flibe blanket was larger than the Flinabe activity in the
DC/Flinabe blanket by a factor of 1.5-2 during few days to few years after
shutdown.  However, the decay heat was much larger in the Flinabe by up to two
orders of magnitude in the time frame of 1 hour-10 years after shutdown. The
Class C \gls{WDR} was estimated for each material. For Flibe, Flinabe and Be
the \gls{WDR} was much lower than unity.  However, the \gls{WDR} for F82H was
\textapprox 0.6-1.3 due to reactions with Mo and Nb present in F82H with
levels of 70 wppm and 4 wppm, respectively. To ensure that F82H qualifies for
shallow land burial, it was suggested to reduce these two impurities to
\textapprox 50 wppm and \textapprox 3 wppm, respectively. The results of this
work was needed to assess safety concerns such as thermal response during
accident conditions and the mobilization of the radiological inventories and
site boundary dose following such
accidents.\citeprod{youssef_activation_2005,abdou_u.s._2005}

\subsection{\acrfull{DCLL} \acrfull{TBM} Design and Analysis}

Various strategic decisions combined with early scoping analysis results in
the selection of a \gls{DCLL} \gls{TBM} concept, requiring more detailed
neutronics design and analysis.\citeprod{wong_overview_2005,
  smolentsev_numerical_2006, wong_overview_2006, ying_overview_2006, wong_overview_2010, wong_progress_2013}


Neutronics calculations were performed to determine the relevant nuclear
performance parameters for the reference US \gls{DCLL} ITER \gls{TBM}. These
include tritium breeding, nuclear heating, radiation damage and
transmutations, and shielding requirements. The neutron wall loading at the
\gls{TBM} is 0.78 MW/m$^2$. The front surface area of the module is 0.8 m$^2$
and the radial depth is 35 cm. The detailed \gls{CAD} model was utilized to
divide the \gls{TBM} into 7 vertical layers and perform calculations for each
with the appropriate radial build. Results for the 7 layers were combined
using their heights to determine the overall integrated parameters for the
\gls{TBM}.

The calculated \gls{TBR} in the \gls{DCLL} \gls{TBM} is only 0.561 because of
the relatively small thickness used. For the planned 3000 pulses per year the
annual tritium production in the \gls{TBM} is 0.97 g. The total nuclear
heating in the \gls{TBM} is 0.574 MW. For the ITER fluence goal of 0.3
MWa/m$^2$, the peak cumulative radiation damage and He production in the FW
are 5.1 dpa and 56 appm, respectively.  The corresponding end-of-life values
for the SiC \gls{FCI} are 4.8 dpa and 409 He appm. The dominant metallic
transmutation product in the \gls{FCI} is Mg that builds up to \textapprox 100
appm at end-of-life of ITER and its impact on electrical and thermal
conductivities of the \gls{FCI} need to be assessed particularly at elevated
fluences in a fusion power plant. We estimated that \textapprox 1.2 m thick
shield is required behind the \gls{DCLL} \gls{TBM} to allow personnel access
for maintenance.\citeprod{sawan_neutronics_2005-1, sawan_neutronics_2009}


Detailed 3-D neutronics calculations were performed for the US \gls{DCLL}
\gls{TBM} to accurately account for the complex geometrical heterogeneity and
impact of source profile and other in-vessel components. The neutronics
calculations were performed directly in the \gls{CAD} model using the DAG-MCNP
code that allows preserving the geometrical details. The 20 cm thick frame
results in neutronics decoupling between the \gls{TBM} and adjacent shield
modules. The \gls{TBM} \gls{CAD} model was inserted in the \gls{CAD} model for
the frame and the integrated \gls{CAD} model was used in the 3-D
analysis. Detailed high-resolution, high-fidelity profiles of the nuclear
parameters were generated using fine mesh tallies. The \gls{TBM}
heterogeneity, exact source profile, and inclusion of the surrounding frame
and other in-vessel components result in lower \gls{TBM} nuclear parameters
compared to the 1-D predictions. The detailed 3-D analysis with the
surrounding massive water-cooled frame and representation of the exact source
and other in-vessel components yields total tritium production in the TBM that
is 45\% lower than the previous 1-D estimate (0.53 g/year). The detailed 3-D
analysis of the \gls{TBM} yields total nuclear heating in the \gls{TBM} that
is 35\% lower than the 1-D estimate of 0.574 MW. The detailed 3-D analysis of
TBM with the surrounding massive water-cooled frame and representation of
exact source and other in-vessel components yields 28\% lower peak dpa rate
and 10\% lower peak He production rate in the \gls{FW} compared to the 1-D
estimates. This is due to the more perpendicular angular distribution of
incident source neutrons in the realistic 3-D configuration and reduced
neutron multiplication and reflection from surrounding frame and other
in-vessel components compared to the 1-D configuration with full coverage with
\gls{DCLL} \gls{TBM}. This work clearly demonstrates the importance of
preserving geometrical details in nuclear analyses of geometrically complex
components in fusion systems.\citeprod{sawan_three-dimensional_2010}


Additional analysis was carried out to consider the consequences of
activation/transmutation of such a design.  The total radioactive inventory in
the \gls{DCLL} \gls{TBM} at shutdown is relatively small (2.44 MCi) and drops
rapidly within a minute to reach a level of \textapprox 0.7 MCi due to the
decay of the Pb-207m isotope. It stays at that level for \textapprox 1 hr and
drops slowly thereafter. The level is \textapprox 0.1 MCi after 1 year and is
\textapprox 0.01 MCi after 10 years. The inventory is almost entirely due to
the activation of the F82H structure in the \gls{TBM}, and in particular, to
the structure in the back breeder channel, the back plate, and the shield. A
few minutes after shutdown, the activation level in the Pb-17Li breeder is
\textapprox 2 orders of magnitude lower than the level in the structure, even
with the inclusion of the activation of the tritium bred while the activation
in the SiC insert is \textapprox 2-6 orders of magnitude lower.

At shutdown, the total decay heat is as low as \textapprox 0.022 MW. After the
decay of the Pb-207m isotope, the total decay heat is attributed mainly to the
structure. The total decay heat after 1 hour, 1 day, 1 year, 10 years, and 100
years is 3.5$times$10$^{-3}$ MW, 1$\times$10$^{-3}$ MW, 1$\times$10$^{-4}$ MW,
2$\times$10$^{-6}$ , and 7$\times$10$^{-10}$ MW, respectively. These are
extremely low values and impose no safety concerns. As is the case for the
radioactive inventory, the decay heat generated in the \gls{FW} is not the
major contributor to the total decay heat. The decay heat generated in the
Pb-17Li breeder is \textapprox 2-3 orders of magnitude lower for all times
after few minutes following shutdown while the attainable level in the SiC
insert is 2-6 orders of magnitude lower than the level in the structure.

The \gls{WDR} depends on the level of the long-term activation. For the F82H
structure, Nb-94, Mn-53, Ni-59, and Nb-91 are the main contributors. In Pb-
17Li, the main contributor is the Pb-205 isotope. The C-14 isotope and the
Be-10 isotope are the main contributors in the SiC insert. The \gls{WDR}
values for F82H structure, the Pb-17Li breeder, and the SiC insert according
to the conservative Fetter limits are 1.3$\times$10$^{-2}$ ,
8.7$\times$10$^{-3}$ , and 2.1$\times$10$^{-4}$ , respectively. They are thus
much lower than unity and therefore these materials are qualified for shallow
land burial according to the Class C limits.\citeprod{youssef_assessment_2005}


\section{Conceptual Design of Fusion Irradiation Facilities}

The \gls{FNSF-AT} (originally namesd the \gls{FDF}) was a small tokamak
concept (R = 2.49 m, r = 0.71 m) with copper magnets with consideration for
both He and water cooling, and for \gls{DCLL} and \gls{HCCB} \gls{FW/B}
systems.  Table \ref{table:FDF-IB} shows the impact of different \gls{IB}
design choices on the \gls{IB} nuclear parameters.  Since the \gls{OB}
responses are not sensitive to the \gls{IB} design choice, Table
\ref{table:FDF-OB} shows the \gls{OB} responses as a function of different
\gls{OB} blanket design options.\citeprod{wong_fusion_2011, wong_fusion_2012}

\begin{table}[htbp]
  \centering
  \caption{Impact of IB design on FDF nuclear parameters}
  \label{table:FDF-IB}
  \begin{tabular}{|l|c|c|c|c|}\hline
    \multirow{2}{*}{\gls{IB} Design Option} & \multicolumn{2}{c|}{He-cooled shield} &  \multicolumn{2}{|c|}{Water-cooled shield} \\\cline{2-5}
    & \gls{DCLL} & \gls{HCCB} & \gls{DCLL} & \gls{HCCB} \\\hline
    \gls{IB} \gls{TBR} & 0.25 & 0.23 & 0.23 & 0.22 \\\hline
OB TBR & 0.93 & 0.83 & 0.92 & 0.83 \\\hline
Total TBR & 1.18 & 1.06 & 1.15 &1.05 \\\hline
Peak He appm in SS VV & 1.68 & 0.80 & 0.31 & 0.29 \\\hline
Peak He appm in FS VV & 0.10 & 0.15 & 0.07 & 0.11 \\\hline
Peak fast neutron fluence in OH coil (10$^{19}$ n/cm$^{2}$) & 19 & 12 & 4.7 & 4.4 \\\hline
Peak organic insulator dose in OH coil (10$^{10}$ Rads) & 21 & 11 & 4.6 & 4.4 \\\hline
Peak fast neutron fluence in TF coil (10$^{19}$ n/cm$^{2}$) & 7.3 & 6.2 & 2.9 & 3.3 \\\hline
Peak organic insulator dose in TF coil (10$^{10}$ Rads) & 2.2 & 1.2 & 4.8 & 4.8 \\\hline
Shield e-fold for organic insulator dose (cm) & 7.6& 7.6& 5.5& 6.3\\\hline
Added shield for organic insulator in OH coil (cm) & \textapprox 23& \textapprox 19& \textapprox 8& \textapprox 10\\\hline
Added shield for organic insulator in TF coil (cm) & \textapprox 23& \textapprox 19& \textapprox 8& \textapprox 10 \\\hline
  \end{tabular}
\end{table}

\begin{table}[htbp]
  \centering
  \caption{Impact of blanket design on \gls{OB} nuclear parameters}
  \label{table:FDF-OB}
  \begin{tabular}{|l|c|c|}\hline
    & \gls{DCLL} & \gls{HCCB} \\\hline
Peak He appm in SS VV & 1.7 & 0.4  \\\hline
Peak He appm in FS VV & 0.10& 0.1 \\\hline
Peak fast neutron fluence (10$^{20}$ n/cm$^{2}$) & 18 & 1.3 \\\hline
Peak organic insulator dose (10$^{11}$ Rads) & 13 & 1.9 \\\hline
Added He-cooled shield for organic insulator (cm) & \textapprox 37& \textapprox 23 \\\hline
Added H$_2$O-cooled shield for organic insulator (cm) & \textapprox 27& \textapprox 19 \\\hline
  \end{tabular}
\end{table}
  
Three-dimensional analysis led to some design modifications to improve the
nuclear performance of this system, with an ultimate \gls{TBR} of 1.09 for the
\gls{HCCB} option and 1.0 for the \gls{DCLL} option.  These low \gls{TBR}
values are, in part, due to an asusmption that no tritium production occurs in
the 16 irradiation ports.  These analyses also provided updated estimates of
peak responses in sensitive components including both \gls{TF} and \gls{PF}
coils.\citeprod{sawan_neutronics_2011}

