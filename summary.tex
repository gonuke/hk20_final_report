\chapter{Summary}

With its origins in the detailed analysis of nuclear performance in fusion
energy systems, this project evolved to incorporate the development of new
tools to improve the predictive nature of that analysis.  As the research
priorities of the \gls{OFES} shifted away from the design and analysis of
nuclear systems, the ongoing benefit of this project has become increasingly
based on those software developments, preparing the predictive capability for
a return to fusion nuclear facility design in the future.

The \gls{APEX} and ITER \gls{TBM} programs included the analysis of a wide
array of \gls{FW/B} options for future fusion experimental facilities and
power plants.  This analysis resulted in valuable insight into the viable
options for achieving tritium self-sufficiency and managing radioactive
wastes, among other criteria.

New software tools were developed to model complex \gls{CAD}-based geoemtries
that are important in fusion energy systems.  Since many of the components
have a primary or secondary role as radiation shields, their geometries
intentionally incorporate features to improve their shielding ability but also
increasing the modeling complexity.  The \gls{DAGMC} toolkit allows the direct
simulation of such geometries and subsequent improvements and extensions also
allow hybrid variance reduction, \gls{SDR} analysis, and the combination of
both.  This capability is garnering increased interest in other application
areas in which complex geometries and/or radiation shielding dominate the
application, including space travel, medical physics and accelerator systems.

As a consumer of nuclear data, we have been monitoring updates and
improvements in the evaluated nuclear cross-section sets, and devising
computational benchmarks to help understand the impact of those changes on
design and analysis of fusion energy systems.  Changes to \gls{FENDL}-2.1 and
\gls{FENDL}3 resulted in modest but non-negligible changes in some nuclear
parameters.

Support for the fusion materials community often results in contributions that
are not easily tracked as a deliverable for this project.  One notable
exception was a comprehensive study of the radiation damage parameters,
including DPA, gas production and transmutation, in SiC composite
materials.  Since this radiation-induced damage produces different
consequences in different components of the composite (matrix vs fibers vs
interface), it warranted a more careful study.  We have also completed an
assessment of the suitability of a fission irradiation environment, notably
\gls{HFIR}, as a substitute for fusion irradiation, but were unable to
complete the same for \gls{SNS} due to a lack of suitable nuclear data at
higher energies.


