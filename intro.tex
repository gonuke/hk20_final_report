\chapter{Introduction}

Over approximately 18 years, this project evolved to focus on a number of
related topics, all tied to the nuclear analysis of fusion energy systems.

For the earliest years, the \gls{UW}'s effort was in support of the \gls{APEX}
study to investigate high power density first wall and blanket systems.  A
variety of design concepts were studied, before this study gave way to a
design effort for a US \gls{TBM} to be installed in ITER.  Simultaneous to
this \gls{TBM} project, nuclear analysis supported the conceptual design of a
number of fusion nuclear science facilities that might fill a role in the path
to fusion energy.  As part of the \gls{APEX} effort, some experimental studies
were performed to better understand the behavior of flowing liquid metals in
magnetic fields.

Beginning in approximately 2005, this project added a component focused on the
development of novel radiation transport software capability in support of the
above nuclear analysis needs.  Specifically, a clear need was identified to
support neutron and photon transport on the complex geometries associated with
\gls{CAD}.  Following the initial development of the \gls{DAGMC} capability,
additional features were added, including unstructured mesh tallies and
multi-physics analysis such as the \gls{R2S} methodology for \gls{SDR}
prediction.

Throughout the project, there were also smaller tasks in support of the fusion
materials community and for the testing of changes to the nuclear data that is
fundamental to this kind of nuclear analysis.

