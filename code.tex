\chapter{Development of Improved Predictive Capability}

Over the life of this project, there was a growing recognition of the impact
of 3-D features on the nuclear performance of fusion energy systems, whether
local material heterogeneity or large scale penetrations that both consume
valuable first wall and breeding space and provide streaming paths for
radiation to the back of the \gls{FW/B} system.  This provided motivation for
the development of new software tools to easily incorporate such 3-D features,
ideally by directly being able to use \gls{CAD} models of the systems in
question

\section{\acrfull{DAGMC}}

The fundamental innovation was the \gls{DAGMC} toolkit that can be integrated
with any Monte Carlo radiation transport tool to allow the direct use of
\gls{CAD}-based geometry descriptions.  While Monte Carlo radiation transport
has long been the preferred way to incorporate complex geometry into an
accurate physics model, faithful representation of many features was a
tedious, time-consuming, and error-prone process.  Manual translation of the
dimensions, surfaces and volumes that might exist in a \gls{CAD} drawing
introduced analysis bottlenecks that often led to approximations.

The \gls{DAGMC}\citeprod{wilson_direct_2017} toolkit was introduced as part of the
open source \gls{MOAB}\citeref{tautges_moab:_2004} library for representing surfaces in a
mesh-like data structure, and relied on capability in the
\gls{CGM}\citeref{tautges_cgm:_2001} library for extracting geometric features.  The
principle scientific contributions developed as part of this toolkit were a
series of accelerations that allowed the Monte Carlo process to proceed on
these complex geometric representations at a speed that was comparable to the
native analytic repesentation.  In particular, the use of \gls{OBB} trees
based on the triangular facets that the solid modeling engine uses to
represent each surface, allows for a logarithmic search for the facet that
will experience the next intersection.  The imprinting and merging capability
provided by \gls{CGM} provides additional benefits, both in reducing the
computational cost of crossing a surface, and in allowing for the implicit
definition of the very complex non-solid space.  The latter feature results in
large benefits in the preparation of a model.\citeprod{wilson_acceleration_2010} It was
demonstrated and validated against a number of standard experimental
benchmarks\citeprod{snouffer_validation_2011,bohm_benchmarking_2011}, as well as during a joint exercise
based on a simplified ITER geometry\citeref{wilson_state---art_2008}, and has since
been used in the majority of 3-D analysis in this and other projects.\citeprod{sawan_three-dimensional_2006,
  sawan_application_2009, sawan_three-dimensional_2010}\citeref{bohm_detailed_2014, bohm_detailed_2015}

Other improvements have focused on robustness, most notably the introduction
of the capability to guarantee so-called watertightness of any analysis
geometry.  This algorithm takes advantage of the topological knowledge
embedded in the \gls{MOAB} representation, and checks every curve in the
geometry for logical consistency of the triangles from different surfaces that
meet on each curve.  If necessary, new triangles are introduced to ensure that
each surface has identical discretization along their common
curves.\citeprod{smith_sealing_2010} Other more subtly improvements were made to the
algorithms to improve the robustness for complex \gls{CAD}-based
geometries.\citeprod{smith_robust_2011}

\subsection{Unstructured Mesh}\label{sec:umesh}

With complex geometries available for the radiation transport phase of nuclear
analysis, it became increasingly important to support the tallying of results
on similarly complex geometries.  Prior capability required the use of a
rectlinear Cartesian mesh that overlays the geometry, with different
cells/materials in many of the hexahedral voxels.  While various
approximations were available for interpreting such results, the most accurate
respresentation requires a mesh that conforms to the shape of each shell,
composed of tetrahera.  Such a capability was added to the \gls{DAGMC} toolkit
and was demonstrated in a number of fusion relevant
problems.\citeref{bohm_detailed_2012}

\section{Hybrid Variance Reduction}\label{sec:hybrid}

Many fusion neutronics analyses focus on predicting the effectiveness of
shields that protect sensitive components or personnel.  Such analyses are
inherently challenging for Monte Carlo radiation transport, even though that
approach is important to properly capture the 3-D features of a model.  By
contrast, deterministic radiation transport methods are much better suited to
shielding problems, but cannot capture effects that arise with detailed
geometric and energy treatments.  One solution is to use these methods
together, with the more approximate deterministic approach providing
acceleration paramters for the Monte Carlo solution.  The most widely used
implementation of such a concept is known as the \gls{CADIS}
method.\citeref{haghighat_monte_2003}

Using the \gls{CADIS} method requires the generation of a rectilinear grid
representation of the complex \gls{CAD}-based geometry.  We implemented a
ray-tracing approach using the same acceleration schemes developed for
radiation transport and demonstrated that such an approach is more efficient
than prior point sampling approaches, particularly when the geometry includes
smaller features.\citeprod{moule_mesh_2009, moule_sampling_2011}

As the geometries grow larger and more complex, the grid representation also
grows, whether covering a larger domain at the same resolution or refining the
resolution to capture smaller features.  Regardless of the motivation, the
size of this grid is limited by two distinct parts of the solution
methodology:
\begin{enumerate}
\item even with the domain decomposed over many parallel processors, there is
  a maximum number of mesh voxels that can fit on each process of the
  deterministic calculation, and
\item there is a maximum size of the acceleration parameters for the Monte Carlo process.
\end{enumerate}
The refinement of the grid for the deterministic calculation should not be
uniform, but rather should focus on resolving important features and
heterogenities.  In addition, the second of these limits is often more
restrictive than the first, requiring a coarsening of the acceleration
parameters from the finest possible mesh achievable in the deterministic
calculation.  This coarsening should also not be uniform, in order to maximize
the utility of the acceleration parameters.  This project supported an effort
automate both the refinement of the mesh for the deterministic calculation and
the subsequent coarsening of the acceleration parameter mesh.  This capability
was demonstrated on a number of different problems.\citeprod{ibrahim_assessment_2014, biondo_accelerating_2015, ibrahim_automatic_2012}\citeref{ibrahim_acceleration_2016}

\section{Shutdown Dose Rates}

Although activation has always been an important consideration in nuclear
analysis of fusion energy systems, radiation doses have typically be estimated
based on 1-D or cruder approximations.  With the ability to represent complex
3-D geometries for radiation transport combined with the licensing of a real
project (ITER), there is a strong motivation for predictive modeling of the
shutdown does rates in a way that takes account of those complex geometries.
One of the primary methodologies for this is known as the \gls{R2S}
methodology because it performs separate radiation transport steps for the
neutrons and the activation photons.  Over time there have been different
implementations with slight differences, but most implementations rely on a
high-fidelity mesh over the entire geometry, with an activation calculation at
each mesh location.  Leveraging the above mentioned capabilities, we have
developed two alternatives for performing an \gls{R2S} analysis: a Cartesian
rectilinear mesh overlayingt he whole geometry or an unstructured mesh (see
Section \ref{sec:umesh}) conforming to each component.  In the former case,
the ray-firing approach described in \ref{sec:hybrid} is necessary to
determine the intial material composition of the activation problem in each
mesh voxel.  In the latter, a novel sampling technique is necessary to
randomly select a position within an arbitrarily shaped
tetrahedron.\citeprod{relson_improved_2013,relson_improved_2013-1} Both of these methodologies were
demonstrated and benchmarked using the ITER shutdown dose rate benchmark
experiment at the Frascati Neutron Generator.\citeprod{biondo_rigorous_2015,biondo_shutdown_2016}

\section{\acrfull{GT-CADIS}}

The final capability that was supported, in part, by this project achieves an
integration of the above-mentioned hybrid acceleration with the \gls{R2S}
\gls{SDR} predictions.  Although various implementations of the \gls{CADIS}
methodology allow for the acceleration of Monte Carlo-based simulations of the
prompt photon dose, it is more challenging to accelerate the simulation of
delayed photon dose.  Where as cross-sections exist and can be measured for
the production of prompt neutrons, delayed neutrons arise from a combination
of transmutation and decay that requires an activation calculation to predict
the exact source.  Under certain assumptions, however, it is possible generate
effective cross-sections for the production of delayed photons.  This allows
for a multi-step approach to the \gls{CADIS} methodology in which acceleration
parameters for the neutron transport step can be determined such that they
ultimately result in an optimum estimate of the delayed photon dose.  The
\gls{SNILB} approximation was shown to be valid for nearly all \gls{MFE}
environments, thus allowing the \gls{GT-CADIS} methodology to be applied to
such systems.  Using this methodology, problem specific pseudo-cross-sections
are evaluated for each material and each shutdown time of interest in a
problem, and used to develop neutron transport acceleration parameters.  This
methodology was demonstrated on the ITER shutdown dose rate benchmark
experiments as well as a characteristic fusion energy
system.\citeprod{biondo_transmutation_2017,biondo_hybrid_2016}

\section{Ongoing Developments}

Three different improvements began in the final months of this project, all
extensions of the \gls{GT-CADIS} capability:
\begin{enumerate}
\item rigorous propagation of statistical error from the neutron transport to
  the delayed photon dose,
\item extension of the pseudo-cross-sections for time-integration to account
  for moving activated components and/or moving dose recipients, and
\item application of the generic multi-step \gls{CADIS} capability to heat
  generation instead of activation.
\end{enumerate} 
